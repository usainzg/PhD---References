\section{Resources}\label{sec:resources}

\subsection{Tools}
\begin{itemize}
    \item TACO: The Tensor Algebra Compiler
    \begin{itemize}
        \item \href{http://groups.csail.mit.edu/commit/}{COMMIT compiler group} led by Professor \href{https://scholar.google.com/citations?hl=en&user=cF6i_goAAAAJ&view_op=list_works&sortby=pubdate}{Saman Amarasinghe} in the CSAIL research lab at MIT.
        \item \href{http://tensor-compiler.org/publications.html}{Publications}: a lot of publications exploring tensor compilers.
        \item \href{http://fredrikbk.com/}{Fredrik Kjolstad}: author of TACO, several works on topics in compilers and programming models. \href{http://fredrikbk.com/publications/kjolstad-thesis.pdf}{His theses} is the best place to start reading about the ideas behind the tensor algebra compiler (taco) and sparse iteration model. 
    \end{itemize}
    \item \href{https://itensor.org/}{ITensor}: High-Performance Tensor Software Inspired By Tensor Diagrams (Julia and C++). 
    \item \href{https://exo-lang.dev/}{Exo Language}: low-level language (and exocompiler) designed to help performance engineers write, optimize, and target high-performance computing kernels onto new hardware accelerators.
    \item \href{https://triton-lang.org/main/index.html}{Triton}: a language and compiler for parallel programming, they revisit “Single Program, Multiple Data” (SPMD) execution models for GPUs, and propose a variant in which programs – rather than threads – are blocked.
    \item \href{https://docs.modular.com/mojo/programming-manual.html}{Mojo language}: write portable code that’s faster than C and seamlessly inter-op with the Python ecosystem. (\href{https://www.nondot.org/sabre/}{Chris Lattner!} - creator of LLVM, Clang, MLIR, etc.).
    \item \href{https://mlir.llvm.org}{MLIR}: is an intermediate representation of a program, not unlike an assembly language, in which a sequential set of instructions operate on in-memory values. MLIR is modular and extensible. MLIR is composed of an ever-growing number of ``dialects'', where each dialect defines operations and optimizations.
    
    \item \href{https://buildit.so/}{BuildIt}: A framework for rapidly developing high-performance Domain Specific Languages (DSLs) with little to no compiler knowledge. (\href{https://github.com/BuildIt-lang/buildit}{github}).
    \begin{itemize}
        \item \href{https://buildit.so/tryit/?sample=einsum}{Einsum-Lang}: Compiler for Einsum-expressions on N-dimensional dense tensors in 300 LoC (CPU and GPU parallel).
        \item \href{https://graphit-lang.org/}{Graphit}: DSL for graph computations that generates fast implementations for algorithms with different performance characteristics running on graphs with different sizes and structures. GraphIt separates what is computed (algorithm) from how it is computed (schedule).
    \end{itemize}
    
\end{itemize}

\subsection{Links}
\begin{itemize}
    \item \href{https://github.com/merrymercy/awesome-tensor-compilers}{Awesome Tensor Compilers}: A list of awesome compiler projects and papers for tensor computation and deep learning.
\end{itemize}

