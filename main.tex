\documentclass[a4paper, 11pt]{article}
%\documentclass[]{book}
\usepackage{unai}

\input config/unai_config

% declare all acronyms
%\newacronym{usg}{USG}{Unai Sainz de la Maza Gamboa}
%\makeglossaries

\title{PhD - References}
\author{Unai Sainz de la Maza Gamboa}
\date{\today}

\begin{document}

\maketitle
\tableofcontents
\clearpage

% \section{Papers}\label{sec:papers}
\subsection{Learning to Optimize Tensor Programs}
In \cite{Chen2018LearningTO}, they introduce a learning-based framework to optimize tensor programs for deep learning workloads, i.e., AutoTVM.
\begin{itemize}
    \item Domain-specific statistical cost models, i.e., machine learning, to guide the search of tensor operator implementations over billions of possible program variants, i.e., multiple variants of low-level code that are logically equivalent.
    \item They try to optimize tensor operator programs for a given hardware platform. Domain-specific features from a given low-level abstract syntax tree (AST) $x$. The features include loop structure information (e.g., memory access count and data reuse ratio) and generic annotations (e.g., vectorization, unrolling, thread binding).
    \item Gradient boosted trees (XGBoost) and TreeGRU (embedding vectors).
\end{itemize}
Prerequisites for automatic code generation: (1) We need to define an exhaustive search space that covers all hardware-aware optimizations in hand-tuned libraries. (2) We need to efficiently find an optimal schedule in the search space. 

For the search space, polyhedral models \cite{Bondhugula2008APA} model the loop domains as integer linear constraint, Halide \cite{RaganKelley2013HalideAL} defines a schedule space using a set of transformation primitives. \textbf{Improving the search space $S_e$ is an important research direction, they do not explore it}, i.e., pick a rich $S_e$ and focus on schedule optimization. They use primitives from a existing code generation framework, TVM, to form the search space. This $S_e$ includes multi-level tiling on each loop axis, loop ordering, shared memory caching for GPUs, and annotations such as unrolling and vectorization. Billions of possible implementations for a single GPU operator.

\subsection{Micro-kernels for portable and efficient matrix multiplication in deep learning}
\uwar{to review}

\subsection{ALCOP: Automatic Load-Compute Pipelining in Deep Learning Compiler for AI-GPUs}
\cite{huang_alcop_2023} overcome three critical obstacles in generating code for pipelining\footnote{the overlap of data loading and computing – is an ideal mechanism for unleashing intra-tile parallelism.}: detection of pipelining-applicable buffers, program transformation for multi-level multi-stage pipelining, and efficient schedule parameter search by incorporating static analysis. Three decoupled and collaborative compilation modules: pipeline buffer detection, pipeline program transformation, and analytical-model guided design space search.
\begin{itemize}
    \item Pipeline buffer detection addresses the workload complexity because it occurs during the scheduling phase when the entire dataflow is visible.
    \item The second module addresses the hardware complexity. During the program transformation stage, the intricate for-loop structure and data movement are revealed and modified. This module utilizes the safety check of the preceding module to execute the robust transformation.
    \item The third module addresses the design space complexity. It happens at the auto-tuning stage, where pipelining and other techniques are co-optimized. This module makes use of the preceding parameterized module.
\end{itemize}

Schedule transformation attaches pipelining primitives to buffer variables in the program, i.e., identify and record ``load-and-use'' structures of the program. Program transformation transforms the program IR (intermediate representation) to implement pipelining. Static analysis guided tuning combines static analytical performance model (see Figure 8) with the existing ML based auto-tuning to choose schedule parameters.

Recently, static analysis has arisen to supplement the standard ML-based schedule tuning, whose cost model lacks hardware knowledge. Tuna \cite{Wang2021TunaAS} builds a performance model for CPU and GPU to replace the ML-based schedule tuning in AutoTVM \cite{Chen2018LearningTO}. Here, they combine ML and static analytical model, with better results than Tuna.

This paper addresses the important need for automatic pipelining in deep learning compilers. Due to the large tiling size required to mitigate bandwidth constraints, inter-tile parallelism is inadequate for achieving high utilization, and intra-tile pipelining becomes essential. Multi-stage, multi-level pipelining.

\subsection{PolyScientist: Automatic Loop Transformations Combined with Microkernels for Optimization of Deep Learning Primitives}
In \cite{Tavarageri2020PolyScientistAL}, they develop a hybrid solution to the development of deep learning kernels that achieves the best of both worlds: the expert coded \textit{microkernels} are utilized for the innermost loops of kernels that exploit the vector register files, and vector units of modern \textbf{CPUs} effectively, and they use the advanced polyhedral compilation technology to automatically tune the outer loops for performance. They design a novel polyhedral model based data reuse algorithm to optimize the outer loops of the kernel. Evaluating on an important class of deep learning primitives namely convolutions, they demonstrate that the approach developed attains the same levels of performance as Intel MKL-DNN, a hand coded deep learning library.
\begin{itemize}
    \item 95\% of all deep learning applications running in the data centers today have a recurring pattern in the inner most loops, namely blocked matrix multiplication \cite{Georganas2019HighPerformanceDL, Hwang2017IndatacenterPA}.
    \item Generated code variants (through code-generator) are then analyzed by our novel data reuse algorithm – \textit{PolyScientist} to characterize their cache behavior. They have developed a relative ranking algorithm which ranks the $n$ variants based on their potential performance. The top $k$ variants are selected and are run on the target hardware and the best performing program version is discovered. From $n$ to $k$ variants actually run on the target architecture, i.e., we suppose that $k \ll n$.
    \item \textit{Relative ranking} heuristic that takes the compiler generated statistics and the system parameters, i.e., cache sizes and ranks the program variants based on their potential performance.
    \item Polyhedral model based cache data reuse analysis to characterize a loop-nest’s behavior with respect to a given cache hierarchy. The analysis computes the various existing data reuses of a program and then for the input cache hierarchy determines which data reuses are exploitable at various levels of cache.
    \item Variants ranking based on performance cost model and DNN-based (tournament based ranking system to assign ranks to the different code versions created – we play each code variant against every other code variant).
    \item TVM, a compiler for deep learning, introduces the concept of \textit{tensorization}, where a unit of computation can be replaced with a microkernel written using hardware intrinsics.
\end{itemize}

\subsection{Domain-Specific Multi-Level IR Rewriting for GPU: The Open Earth Compiler for GPU-accelerated Climate Simulation}
In \cite{gysi_domain-specific_2021} practical case study implementing a domain-specific compiler for weather and climate modeling. They propose to design DSL compilers using \textit{multi-level IR rewriting}. This approach is a combination of (a) intermediate representations (IR) based on Static Single Assignment form
(SSA) \cite{Rosen1988GlobalVN}, (b) operations with high-level semantics, and (c) progressive lowering, which provides an effective framework for reusable domain-specific high-performance code generation. SSA-based IRs allow us to reuse optimizations from general-purpose compilers.

\begin{itemize}
    \item Multi-level rewriting instantiates a hierarchy of dialects (IRs), lowers programs level-by-level, and performs code transformations at the most suitable level.
    \item A set of MLIR dialects, i.e., collections of domain-specific operations and transformations, and conversions between them.
    \item Multi-level IR rewriting and the associated design principles is a promising approach to rapidly design and deploy domain-specific compilers that can take advantage of reusable components of the MLIR ecosystem.
\end{itemize}

\subsection{TensorIR: An Abstraction for Automatic Tensorized Program Optimization}
In \cite{Feng2022TensorIRAA}, they present a compiler abstraction for optimizing programs with these tensor computation primitives. Generalizes the loop nest representation used in existing machine learning compilers. TensorIR compilation automatically uses the tensor computation primitives for given hardware backends. Challenges bringing automatic program optimization to tensorized programs:
\begin{itemize}
    \item \textit{Abstraction for Tensorized Programs}: we need an abstraction that can pragmatically capture possible equivalent tensorized computations for a given machine learning operator. Notably, the abstraction needs to represent multi-dimensional memory accesses, threading hierarchies, and tensorized computation primitives from different hardware backends.
    \item \textit{Large Design Space of Possible Tensorized Program Optimizations}: we need an effective way to find an optimized tensorized program for a given search space.
\end{itemize}
TensorIR parts:
\begin{itemize}
    \item How to write? $\rightarrow$ TVMScript (python-AST based syntax, i.e., parsed by python-AST) with multi-dim buffers, loop nests and computational blocks (vectorized/tensorized computation). A block contains four major parts: outside loop nesting, block iterator domain, producer/consumer dependency relations and the block body (only indexed by block iterator).
    \item How to optimize? $\rightarrow$ Interactive Schedule on Tensorized Body (tree manipulation, block isolation, easy tensorization, interactivity).
    \item How to customize? $\rightarrow$ Decoupled Primitives
\end{itemize}

\subsection{PowerFusion: A Tensor Compiler with Explicit Data Movement Description and Instruction-level Graph IR}
In \cite{ma_powerfusion_2023}, they present a tensor compiler that can generate high-performance code for memory-intensive operators by considering both computation and data movement optimizations. Represent a DNN program using GIR (instruction-level graph IR), which includes primitives indicating its computation, data movement, and parallel strategies. This information will be further composed as an instruction-level dataflow graph to perform holistic optimizations by searching different memory access patterns and computation operations, and generating memory-efficient code on different hardware.
\begin{itemize}
    \item Explicit description of data movement in the IR. Compiler needs to analyze data dependencies at a fine granularity to reuse data using high-level memory hierarchy.
    \item TVM and Ansor represent a tensor program with an abstraction of compute and schedule, inspired by Halide. They first convert DNN models into a loop-based IR (compute) and apply series of optimizations (schedule) to transform DNN programs to find better performance. Efficient code tailored to specific architectures.
    \item Three challenges for memory-intensive programs: implicit data movement representation (implicit through the schedule, challenging to assess memory performance during optimization), schedule search order (generation order prevents the application of different memory access patterns), and coarse-grained dependence analysis (limits dependence analysis only to loops).
    \item They represent tensor programs using GIR, three primitives: parallel (parallel strategies for a given computation), and the computation and data movement primitives, which they call gOperators, are constructed as nodes of an instruction-level dataflow graph, called GIR-Graph, and the edge between them represents the data on certain memory hierarchy and also their dependence. By explicitly expressing memory access operations and instruction-level dependence, they can search different memory access pattern for multiple computing operations.
\end{itemize}
% \cleardoublepage
% \section{Resources}\label{sec:resources}

\subsection{Tools}
\begin{itemize}
    \item TACO: The Tensor Algebra Compiler
    \begin{itemize}
        \item \href{http://groups.csail.mit.edu/commit/}{COMMIT compiler group} led by Professor \href{https://scholar.google.com/citations?hl=en&user=cF6i_goAAAAJ&view_op=list_works&sortby=pubdate}{Saman Amarasinghe} in the CSAIL research lab at MIT.
        \item \href{http://tensor-compiler.org/publications.html}{Publications}: a lot of publications exploring tensor compilers.
        \item \href{http://fredrikbk.com/}{Fredrik Kjolstad}: author of TACO, several works on topics in compilers and programming models. \href{http://fredrikbk.com/publications/kjolstad-thesis.pdf}{His theses} is the best place to start reading about the ideas behind the tensor algebra compiler (taco) and sparse iteration model. 
    \end{itemize}
    \item \href{https://itensor.org/}{ITensor}: High-Performance Tensor Software Inspired By Tensor Diagrams (Julia and C++). 
    \item \href{https://exo-lang.dev/}{Exo Language}: low-level language (and exocompiler) designed to help performance engineers write, optimize, and target high-performance computing kernels onto new hardware accelerators.
    \item \href{https://triton-lang.org/main/index.html}{Triton}: a language and compiler for parallel programming, they revisit “Single Program, Multiple Data” (SPMD) execution models for GPUs, and propose a variant in which programs – rather than threads – are blocked.
    \item \href{https://docs.modular.com/mojo/programming-manual.html}{Mojo language}: write portable code that’s faster than C and seamlessly inter-op with the Python ecosystem. (\href{https://www.nondot.org/sabre/}{Chris Lattner!} - creator of LLVM, Clang, MLIR, etc.).
    \item \href{https://mlir.llvm.org}{MLIR}: is an intermediate representation of a program, not unlike an assembly language, in which a sequential set of instructions operate on in-memory values. MLIR is modular and extensible. MLIR is composed of an ever-growing number of ``dialects'', where each dialect defines operations and optimizations.
    
    \item \href{https://buildit.so/}{BuildIt}: A framework for rapidly developing high-performance Domain Specific Languages (DSLs) with little to no compiler knowledge. (\href{https://github.com/BuildIt-lang/buildit}{github}).
    \begin{itemize}
        \item \href{https://buildit.so/tryit/?sample=einsum}{Einsum-Lang}: Compiler for Einsum-expressions on N-dimensional dense tensors in 300 LoC (CPU and GPU parallel).
        \item \href{https://graphit-lang.org/}{Graphit}: DSL for graph computations that generates fast implementations for algorithms with different performance characteristics running on graphs with different sizes and structures. GraphIt separates what is computed (algorithm) from how it is computed (schedule).
    \end{itemize}
    
\end{itemize}

\subsection{Links}
\begin{itemize}
    \item \href{https://github.com/merrymercy/awesome-tensor-compilers}{Awesome Tensor Compilers}: A list of awesome compiler projects and papers for tensor computation and deep learning.
\end{itemize}


% \cleardoublepage
% \section{Polyhedral}\label{sec:polly}
\subsection{A practical tile size selection model for affine loop nests}

\subsection{Maximal Atomic irRedundant Sets: a Usage-based Dataflow Partitioning Algorithm}

\subsection{Polygeist: Affine C in MLIR}

\subsection{Automatic multi-dimensional pipelining for high-level synthesis of dataflow accelerators}

\subsection{Scalable Polyhedral Compilation, Syntax vs. Semantics: 1–0 in the First Round}
In \cite{48842}, a family of techniques called offline statement clustering, which integrates transparently into the flow of a state-of-the-art polyhedral compiler and can reduce the scheduling time by a factor of 6 without inducing a significant loss in optimization opportunities is introduced.

\subsection{Polyhedral Compilation for Racetrack Memories}
In \cite{khan2020polyhedral}, they present the first automatic compilation framework that optimizes static loop programs over arrays for linear-latency memories, extending the polyhedral compilation framework Polly to generate code that maximizes accesses to the same or consecutive locations, thereby minimizing the number of shifts.

\subsection{Polyhedral Compilation for Multi-dimensional Stream Processing}
In \cite{leben2019polyhedral}, they present a method that involves a novel polyhedral schedule transformation called periodic tiling that enables efficient execution of programming languages with unbounded recurrence equations, as well as optimization of existing languages from which this form can be derived.

\subsection{An Autotuning Framework for Scalable Execution of Tiled Code via Iterative Polyhedral Compilation}

\subsection{Employing polyhedral methods to optimize stencils on FPGAs with stencil-specific caches, data reuse, and wide data bursts}

\subsection{Modelling linear algebra kernels as polyhedral volume operations}

\subsection{Automated Partitioning of Data-Parallel Kernels using Polyhedral Compilation}

\subsection{Superloop Scheduling: Loop Optimization via Direct Statement Instance Reordering}
In \cite{bastoul2023superloop}, they propose a different approach to affine scheduling construction called superloop scheduling. A 7-step loop optimization process:
\begin{itemize}
    \item Iteration space bounding: limit the number of iterations and replaces the parameters with known values.
    \item Full loop unrolling: transform the code to a sequence of statement instances.
    \item Statement instance reordering: transform the unrolled code via direct reasoning and manipulation of the statement instances. Parallel block extraction and internal reordering to enable vectorization. Also, basic block level techniques such as superword level parallelisation \cite{Mendis_2018} to enable possibly unprecedented loop vectorization opportunities. Reordering policies should include constraints or mechanisms to favor some regularity when possible.
    \item Nested loop recognition (NLR): recover loops in a fast and incremental way \cite{ketterlin2008prediction} from a trace comprised of tagged vectors of numbers. NLR is able to recover arbitrarily deep and/or complex affine loop nests from their traces.
    \item Affine scheduling reconstruction: build an affine scheduling expression from the loop structure and the mapping information present in NLR’s output. Also, attempt to recover parameters, e.g. through pattern-matching.
    \item Legality check: verify the scheduling correctness on the original input code using the Candl tool, and may assess its properties such as parallel and vector dimensions.
    \item Code generation: produce the optimized code that implements the scheduling using, e.g., CLooG \cite{bastoul2004code}. The optimization is different than both Feautrier \cite{feautrier1992some} (which favors k, i, j version with internal parallelism) and Pluto without tiling \cite{bondhugula2008practical} (which splits S0 and S1 in two loop nests to enable i, k, j order for S1, or uses less CPU-efficient i, j, k).
\end{itemize}
Two families of techniques for affine scheduling construction:
\begin{itemize}
    \item Compute the scheduling by solving systems of affine constraints: proposed by Feautrier \cite{feautrier1992some} and has set the ground for many later techniques, the Pluto algorithm designed by Bondhugula et al. to extract outermost parallelism, data locality and tilable loops \cite{bondhugula2008practical}, and a number of variants that differ in the way the affine constraint system is built.
    \item Build the affine scheduling by composition of basic primitives: proposed by Kelly and Pugh \cite{kelly1998framework} and improved by many authors, e.g., Baghdadi et al. who propose a rich scheduling language for the Tiramisu framework \cite{baghdadi2019tiramisu}. 
\end{itemize}
However, in this work, they present the seed for a new family that builds affine scheduling from finest-grain statement instance reordering and loop reconstruction. Which could have a significant potential for exploiting custom vector instructions.

\textbf{Important note:} presented at IMPACT 2023, a short (working) paper.

\subsection{Pipelined Multithreading Generation in a Polyhedral Compiler}



\section{Compiler-based distributed computing techniques (\uwar{thesis title?})}
\subsection{Summary}
Compute-intensive applications running on clusters of shared-memory computers typically utilize OpenMP and MPI for implementation. These applications are challenging to program, debug, and maintain. Additionally, achieving performance portability is often limited, requiring several program transformations at multiple levels of the software and hardware stack to leverage features like parallelism and granularity. Finally, specific characteristics of the target system, such as network topology and node capacity, must be considered to achieve high-performing code.

Performance portability and user productivity are major concerns for modern compilers, as significant code reorganization and restructuring are often required to adapt applications to the resources of a computer cluster. Consequently, polyhedral auto-transformation frameworks have garnered significant interest in general-purpose compilation due to their capability to identify and implement complex loop transformations that extract high performance from modern architectures. These frameworks automatically identify loop transformations that enhance locality, parallelism, minimize latency, or achieve a combination of these improvements.

In this thesis, we try to improve the applicability and profitability of polyhedral-model-based techniques for distributed computing challenges, such as scheduling and communication optimization. Specifically, we enrich the polyhedral program representation with domain-specific knowledge from distributed computing systems, e.g., network topology or nodes capacity.

\subsection{Background}
The polyhedral model is a mathematical representation of programs that simplifies both analysis and restructuring. It provides a compact and expressive way to describe parallelization and optimization problems. Unlike operational or syntactic representations, this model is closer to program execution because it operates on individual statement iterations or instances. For each instance, the optimizing algorithm computes a mapping to determine the execution time (time mapping or scheduling) and/or the processor on which it will run (space mapping or placement) \cite{}.

The pioneering work of the polyhedral model is attributed to Karp et al. on systems of uniform recurrence equations \cite{}. With the advancement of polyhedral compilation, integer polyhedra \cite{} and Presburger relations \cite{} were introduced to enhance expressiveness and flexibility. Polyhedral compilation can either be integrated as a building block into general-purpose compilers, such as Polly in LLVM \cite{} and affine in MLIR \cite{}, or serve as a source-to-source translator like Pluto \cite{} and PPCG \cite{}. This flexibility and compatibility enable the construction of a well-defined compilation workflow and significantly extend its application domain.

Polyhedral compilation typically comprises three steps: modeling, transformation, and code generation. In the modeling step, a mathematical representation of the program is constructed using integer polyhedra and Presburger relations to capture the program's computational patterns. The transformation step then applies various optimizations, such as loop transformations, to enhance parallelism and data locality. Finally, the code generation step translates the optimized polyhedral model back into executable code.

The polyhedral model has proven useful for transforming and generating parallel programs from sequential codes featuring affine loop nests \cite{}. Additionally, it facilitates automatic code generation for distributed-memory platforms \cite{}. The model's dependence analysis supports code generation by (1) identifying values requiring communication across processes, (2) managing data packing and unpacking, and (3) executing necessary communication operations. Griebl et al. pioneered the use of the polyhedral model on distributed-memory systems \cite{}, but their approach suffers from significant redundancy in communication operations. Existing methods for calculating communication in distributed memory, such as those discussed in \cite{}, focus on sequences of nested loops with regular (affine) accesses, known as affine loop nests. These methods involve loop transformation, tiling, and parallelization. However, they often overlook characteristics specific to the target distributed system, such as heterogeneity of nodes, network topology, and node capacities.

\uwar{More?}

\subsection{Objectives}
\uwar{ok? Plantilla?}
Based on the issues outlined in the previous section, the research question to be addressed in this Ph.D. thesis is: ...
\begin{itemize}
    \item Is it possible to improve the application of polyhedral compilation to distributed systems by means of specific information such as topology?
    \item Can we improve the techniques used in distributed systems with the help of compilers?
\end{itemize}

\subsection{Methodology}
\uwar{ok? Plantilla?}
The research methodology employed in this thesis is based on an experimental approach and a software engineering method encompassing four stages: observing existing solutions, proposing improved solutions, developing the proposed solutions, and measuring and analyzing the outcomes \cite{adrion1993research}. This iterative methodology is repeated to refine the solutions. It parallels the stages of the classical scientific method: proposing a question, formulating a hypothesis, making predictions, and validating the hypothesis.

\begin{enumerate}
    \item Observe existing solutions: this is an exploratory phase where the literature and state-of-the-art tools related to our research field are studied to identify potential improvements.
    \item Propose better solutions: this phase focuses on designing and analyzing improved solutions, aiming to surpass the limitations of previous proposals or enhance existing methods from the literature.
    \item Build or develop the solutions: in this phase, we concentrate on constructing a prototype to demonstrate the feasibility of the solution.
    \item Measure and analyze the new solutions: finally, the implemented prototypes are empirically evaluated and compared with various alternatives. The goal of this evaluation is to validate the proposed solution and confirm that the problems identified in the initial step have been resolved.
\end{enumerate}

\subsection{Work plan}
\uwar{Plantilla?}
\subsubsection{Part I: Topology-aware distributed data communication optimization?}
\subsubsection{Part II: ...}

\subsection{Scientific or social interest}

\unote{Requisitos:} Memoria científico–técnica en formato normalizado del proyecto de tesis doctoral firmado por la persona directora del mismo, y codirectora si la hubiere, y cuyo contenido será el siguiente: título del proyecto, resumen, antecedentes, objetivos, metodología y plan de trabajo, interés científico y/o social y la bibliografía más destacada (máximo 5 páginas). Además de este contenido, dicha memoria deberá recoger expresamente lo siguiente: Si es o no esperable que los objetivos del proyecto de Tesis Doctoral y su defensa se alcancen por la persona solicitante en 3 años de ayuda con dedicación completa.


\section{Polyhedral papers}\label{sec:polly_papers}
\subsection{Topics (taken from IMPACT CFP)}
\begin{itemize}
    \item Program optimization (automatic parallelization, tiling, etc.)
    \item Code generation.
    \item Data/communication management on GPUs, accelerators and distributed systems.
    \item Hardware/high-level synthesis for affine programs.
    \item Static analysis.
    \item Program verification.
    \item Model checking.
    \item Theoretical foundations of the polyhedral model.
    \item Extensions of the polyhedral model.
    \item Scalability and robustness of polyhedral compilation techniques.
\end{itemize}

\subsection{Courses}
\begin{itemize}
    \item Polyhedral seminar (C. Alias): \href{https://gitlab.inria.fr/alias/polyhedral-seminar}{Link}
    \item UCLA CS33/CS31 (LN. Pouchet): \href{https://web.cs.ucla.edu/~pouchet/index.html#lectures}{Link}
    \item Presburger Formulas and Polyhedral Compilation Tutorial (S. Verdoolaege): \href{https://lirias.kuleuven.be/bitstream/123456789/523109/3/polycomp-tutorial-v0.02.pdf}{Link}
    \item Static Analysis and Optimizing Compilers (C. Alias): \href{https://gitlab.inria.fr/alias/cours-m2-cr14}{Link}
    \item Polyhedral Compilation as a Design Pattern for Compilers (A.Cohen): \href{https://www.youtube.com/watch?v=mt6pIpt5Wk0}{Link 1} and \href{https://www.youtube.com/watch?v=3TNT5rFVTUY}{Link 2}
\end{itemize}

\subsection{The Polyhedral Model Is More Widely Applicable Than You Think}
\cite{Benabderrahmane2010ThePM}

\subsection{Modelling linear algebra kernels as polyhedral volume operations}
In \cite{friebel2022modelling}, programs are represented as sequences of operations on indexed volumes characterized by their element-wise dependencies, efficiently implementing compound kernels by partitioning and specializing over index domains.
\begin{itemize}
    \item They generalize over arrays and define a volume as an aggregate value comprising indexed element values (scalars). Volume is characterized by an index domain, which is a polyhedron of all element indices.
    \item Volumes permit arbitrary polyhedral index domains and are intended to model the structural properties of data, as well as memory when needed.
    \item They perform dynamic shape inference of input/output/intermediate volumes. They use affine pattern matching to identify regions of interest (sparse regions in a volume) and specialize over them. They separate computation from memory (operational semantics based on general affine volumes, not strictly hyperrectangular ones).
    \item The proposed model can be used on the MLIR linalg abstraction to generalize tiling and other partitioning transformations to arbitrary tensor programs.
\end{itemize}
The main idea is to detect sparsity and other structural properties, and act on them by partitioning computations. State-of-the-art compilers struggle to separate the different regions of the program, which may lead to sub-optimal performance. However, their model can conveniently identify all kernel regions, i.e., dense, sparse, constant, and corners. This empowers specializing over these regions, e.g., by using sparse compiler optimizations for sparse regions or offloading them to a sparsity-aware hardware target and mapping dense regions to external library calls.

\subsection{ParameTrick: Coefficient Generalization for Faster Polyhedral Scheduling}
In \cite{consolaro24-parametrick}, 

\subsection{A practical tile size selection model for affine loop nests}

\subsection{Maximal Atomic irRedundant Sets: a Usage-based Dataflow Partitioning Algorithm}

\subsection{A polyhedral compilation framework for loops with dynamic data-dependent bounds}

\subsection{Generating SIMD Instructions for Cerebras CS-1 using Polyhedral Compilation Techniques}
In \cite{verdoolaege2020generating}, they use polyhedral compilation to generate SIMD instructions for the cerebras cs-1 computing system. More precisely, they use polyhedral operations such as variable compression and fixed-sized box hull.
\begin{itemize}
    \item SIMD engine that can mimic a rectangular loop nest of depth at most four. For this, is important to represent the set of computation instances as a rectangular domain.
    \item They rely on \textit{isl} library for manipulating sets of integer tuples described by Presburger formulas (a first order logic formula involving affine expressions, equality and the less-than-or-equal relation). 
    \item Variable compression exploits the equality constraints in the description of a set to obtain a set with the same number of points, but of a lower dimensionality.
    \item Fixed-size box hull operation examines the constraints of a binary relation to find an overapproximation where the range can be described as a rectangular box with fixed size and an offset that depends on the domain variables and the symbolic constants.
    \item Target architecture: MPPA (Massively Parallel Processor Array), consisting of a 2-dimensional grid of PEs (processing elements) that communicate with their nearest neighbors in the four cardinal directions. Memory distributed overt the PEs and hardware performs dataflow scheduling. (not much public details).
\end{itemize}
The main idea is to expose rectangle sets of operations that can be mapped to the advanced SIMD operations of the Cerebras CS-1 architecture.

\subsection{Polygeist: Affine C in MLIR}

\subsection{Automatic multi-dimensional pipelining for high-level synthesis of dataflow accelerators}

\subsection{Tile size selection of affine programs for GPGPUs using polyhedral cross-compilation}
In \cite{Abdelaal2021TileSS}, they use Integer Linear Programming (ILP) constraints and objectives in a cross-compiler fashion to faithfully and effectively mimic the transformations applied in a polyhedral GPU compiler (PPCG) to compute resource-conscious tile sizes for affine programs.

\subsection{PolySA: Polyhedral-Based Systolic Array Auto-Compilation}
In \cite{Cong2018PolySAPS}, they present PolySA, the first fully automated compilation framework for generating high performance systolic array architectures on the FPGA that leverages recent advances in high-level synthesis and can generate optimal designs in one hour with performance comparable to state-of-the-art manual designs.

\subsection{AutoSA: A Polyhedral Compiler for High-Performance Systolic Arrays on FPGA}
In \cite{Wang2021AutoSAAP}, they present AutoSA, an end-to-end compilation framework for generating systolic arrays on FPGA, based on the polyhedral framework, and further incorporates a set of optimizations on different dimensions to boost performance.

\subsection{Scalable Polyhedral Compilation, Syntax vs. Semantics: 1–0 in the First Round}
In \cite{48842}, a family of techniques is introduced called offline statement clustering, which integrates transparently into the flow of a state-of-the-art polyhedral compiler and can reduce the scheduling time by a factor of 6 without inducing a significant loss in optimization opportunities.

\subsection{Polyhedral Compilation for Racetrack Memories}
In \cite{khan2020polyhedral}, they present the first automatic compilation framework that optimizes static loop programs over arrays for linear-latency memories, extending the polyhedral compilation framework Polly to generate code that maximizes accesses to the same or consecutive locations, thereby minimizing the number of shifts.

\subsection{Polyhedral Compilation for Multi-dimensional Stream Processing}
In \cite{leben2019polyhedral}, they present a method that involves a novel polyhedral schedule transformation called periodic tiling that enables efficient execution of programming languages with unbounded recurrence equations, as well as optimization of existing languages from which this form can be derived.

\subsection{An Autotuning Framework for Scalable Execution of Tiled Code via Iterative Polyhedral Compilation}

\subsection{Employing polyhedral methods to optimize stencils on FPGAs with stencil-specific caches, data reuse, and wide data bursts}
In \cite{mayer24-fpgas}, polyhedral methods are used to generate stencil-specific cache-structures of the right sizes on the FPGA and to fill and flush them efficiently with wide bursts during stencil execution. They derive the appropriate directives and code restructurings from stencil codes so that the FPGA compiler generates fast stencil hardware.
\begin{itemize}
    \item From previous attempts: HLS cannot turn the tiled loop into hardware pipelines (as the loop boundaries and increments are too complex for the HLS to detect patterns). On CPUs or GPUs tiled loops run faster since their data locality exploits the built-in cache hierarchy, however, on FPGAs there are no caches, there is no benefit from tiling alone. In general, the HLS generates from the tiled loop a hardware block (kernel) with connections to an FPGA bus through which the kernel directly talks to the DDR.
    \item Polyhedral methods are used to understand the access patterns we can statically determine (a) the stencil-specific best cache sizes and (b) pre-load exactly what is needed and spill what is no longer needed. But the polyhedral method can also (c) help improve the burst behavior. They also (d) inline the cache functionality, i.e., use buffer arrays and load operations within the stencil.
\end{itemize}
This work uses loop tiling to improve locality, adds buffers and directives to generate inlined cache-like hardware circuits that exploit that locality, and calculates a data-shipment that uses wide bursts to fill these caches, i.e., a polyhedral-based code generator for FPGAs.

\subsection{Modelling linear algebra kernels as polyhedral volume operations}

\subsection{Automated Partitioning of Data-Parallel Kernels using Polyhedral Compilation}

\subsection{Superloop Scheduling: Loop Optimization via Direct Statement Instance Reordering}
In \cite{bastoul2023superloop}, they propose a different approach to affine scheduling construction called superloop scheduling. A 7-step loop optimization process:
\begin{itemize}
    \item Iteration space bounding: limit the number of iterations and replaces the parameters with known values.
    \item Full loop unrolling: transform the code to a sequence of statement instances.
    \item Statement instance reordering: transform the unrolled code via direct reasoning and manipulation of the statement instances. Parallel block extraction and internal reordering to enable vectorization. Also, basic block level techniques such as superword level parallelisation \cite{Mendis_2018} to enable possibly unprecedented loop vectorization opportunities. Reordering policies should include constraints or mechanisms to favor some regularity when possible.
    \item Nested loop recognition (NLR): recover loops in a fast and incremental way \cite{ketterlin2008prediction} from a trace comprised of tagged vectors of numbers. NLR is able to recover arbitrarily deep and/or complex affine loop nests from their traces.
    \item Affine scheduling reconstruction: build an affine scheduling expression from the loop structure and the mapping information present in NLR’s output. Also, attempt to recover parameters, e.g. through pattern-matching.
    \item Legality check: verify the scheduling correctness on the original input code using the Candl tool, and may assess its properties such as parallel and vector dimensions.
    \item Code generation: produce the optimized code that implements the scheduling using, e.g., CLooG \cite{bastoul2004code}. The optimization is different than both Feautrier \cite{feautrier1992some} (which favors k, i, j version with internal parallelism) and Pluto without tiling \cite{bondhugula2008practical} (which splits S0 and S1 in two loop nests to enable i, k, j order for S1, or uses less CPU-efficient i, j, k).
\end{itemize}
Two families of techniques for affine scheduling construction:
\begin{itemize}
    \item Compute the scheduling by solving systems of affine constraints: proposed by Feautrier \cite{feautrier1992some} and has set the ground for many later techniques, the Pluto algorithm designed by Bondhugula et al. to extract outermost parallelism, data locality and tilable loops \cite{bondhugula2008practical}, and a number of variants that differ in the way the affine constraint system is built.
    \item Build the affine scheduling by composition of basic primitives: proposed by Kelly and Pugh \cite{kelly1998framework} and improved by many authors, e.g., Baghdadi et al. who propose a rich scheduling language for the Tiramisu framework \cite{baghdadi2019tiramisu}. 
\end{itemize}
However, in this work, they present the seed for a new family that builds affine scheduling from finest-grain statement instance reordering and loop reconstruction. Which could have a significant potential for exploiting custom vector instructions.

\textbf{Important note:} presented at IMPACT 2023, a short (working) paper.

\subsection{Pipelined Multithreading Generation in a Polyhedral Compiler}

\subsection{Compiling affine loop nests for distributed-memory parallel architectures}
\cite{bondhugulaCompilingAffineLoop2013}

\subsection{Dynamic memory access monitoring based on tagged memory}
\cite{dathathriDynamicMemoryAccess2013}

\subsection{Effective automatic computation placement and data allocation for parallelization of regular programs}
\cite{reddyEffectiveAutomaticComputation2014}

\cleardoublepage
\bibliographystyle{unsrt}
\bibliography{refs, references}

%\printglossary[type=\acronymtype]
\end{document}